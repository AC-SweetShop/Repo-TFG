
% Default to the notebook output style

    


% Inherit from the specified cell style.




    
\documentclass[11pt]{article}

    
    
    \usepackage[T1]{fontenc}
    % Nicer default font (+ math font) than Computer Modern for most use cases
    \usepackage{mathpazo}

    % Basic figure setup, for now with no caption control since it's done
    % automatically by Pandoc (which extracts ![](path) syntax from Markdown).
    \usepackage{graphicx}
    % We will generate all images so they have a width \maxwidth. This means
    % that they will get their normal width if they fit onto the page, but
    % are scaled down if they would overflow the margins.
    \makeatletter
    \def\maxwidth{\ifdim\Gin@nat@width>\linewidth\linewidth
    \else\Gin@nat@width\fi}
    \makeatother
    \let\Oldincludegraphics\includegraphics
    % Set max figure width to be 80% of text width, for now hardcoded.
    \renewcommand{\includegraphics}[1]{\Oldincludegraphics[width=.8\maxwidth]{#1}}
    % Ensure that by default, figures have no caption (until we provide a
    % proper Figure object with a Caption API and a way to capture that
    % in the conversion process - todo).
    \usepackage{caption}
    \DeclareCaptionLabelFormat{nolabel}{}
    \captionsetup{labelformat=nolabel}

    \usepackage{adjustbox} % Used to constrain images to a maximum size 
    \usepackage{xcolor} % Allow colors to be defined
    \usepackage{enumerate} % Needed for markdown enumerations to work
    \usepackage{geometry} % Used to adjust the document margins
    \usepackage{amsmath} % Equations
    \usepackage{amssymb} % Equations
    \usepackage{textcomp} % defines textquotesingle
    % Hack from http://tex.stackexchange.com/a/47451/13684:
    \AtBeginDocument{%
        \def\PYZsq{\textquotesingle}% Upright quotes in Pygmentized code
    }
    \usepackage{upquote} % Upright quotes for verbatim code
    \usepackage{eurosym} % defines \euro
    \usepackage[mathletters]{ucs} % Extended unicode (utf-8) support
    \usepackage[utf8x]{inputenc} % Allow utf-8 characters in the tex document
    \usepackage{fancyvrb} % verbatim replacement that allows latex
    \usepackage{grffile} % extends the file name processing of package graphics 
                         % to support a larger range 
    % The hyperref package gives us a pdf with properly built
    % internal navigation ('pdf bookmarks' for the table of contents,
    % internal cross-reference links, web links for URLs, etc.)
    \usepackage{hyperref}
    \usepackage{longtable} % longtable support required by pandoc >1.10
    \usepackage{booktabs}  % table support for pandoc > 1.12.2
    \usepackage[inline]{enumitem} % IRkernel/repr support (it uses the enumerate* environment)
    \usepackage[normalem]{ulem} % ulem is needed to support strikethroughs (\sout)
                                % normalem makes italics be italics, not underlines
    

    
    
    % Colors for the hyperref package
    \definecolor{urlcolor}{rgb}{0,.145,.698}
    \definecolor{linkcolor}{rgb}{.71,0.21,0.01}
    \definecolor{citecolor}{rgb}{.12,.54,.11}

    % ANSI colors
    \definecolor{ansi-black}{HTML}{3E424D}
    \definecolor{ansi-black-intense}{HTML}{282C36}
    \definecolor{ansi-red}{HTML}{E75C58}
    \definecolor{ansi-red-intense}{HTML}{B22B31}
    \definecolor{ansi-green}{HTML}{00A250}
    \definecolor{ansi-green-intense}{HTML}{007427}
    \definecolor{ansi-yellow}{HTML}{DDB62B}
    \definecolor{ansi-yellow-intense}{HTML}{B27D12}
    \definecolor{ansi-blue}{HTML}{208FFB}
    \definecolor{ansi-blue-intense}{HTML}{0065CA}
    \definecolor{ansi-magenta}{HTML}{D160C4}
    \definecolor{ansi-magenta-intense}{HTML}{A03196}
    \definecolor{ansi-cyan}{HTML}{60C6C8}
    \definecolor{ansi-cyan-intense}{HTML}{258F8F}
    \definecolor{ansi-white}{HTML}{C5C1B4}
    \definecolor{ansi-white-intense}{HTML}{A1A6B2}

    % commands and environments needed by pandoc snippets
    % extracted from the output of `pandoc -s`
    \providecommand{\tightlist}{%
      \setlength{\itemsep}{0pt}\setlength{\parskip}{0pt}}
    \DefineVerbatimEnvironment{Highlighting}{Verbatim}{commandchars=\\\{\}}
    % Add ',fontsize=\small' for more characters per line
    \newenvironment{Shaded}{}{}
    \newcommand{\KeywordTok}[1]{\textcolor[rgb]{0.00,0.44,0.13}{\textbf{{#1}}}}
    \newcommand{\DataTypeTok}[1]{\textcolor[rgb]{0.56,0.13,0.00}{{#1}}}
    \newcommand{\DecValTok}[1]{\textcolor[rgb]{0.25,0.63,0.44}{{#1}}}
    \newcommand{\BaseNTok}[1]{\textcolor[rgb]{0.25,0.63,0.44}{{#1}}}
    \newcommand{\FloatTok}[1]{\textcolor[rgb]{0.25,0.63,0.44}{{#1}}}
    \newcommand{\CharTok}[1]{\textcolor[rgb]{0.25,0.44,0.63}{{#1}}}
    \newcommand{\StringTok}[1]{\textcolor[rgb]{0.25,0.44,0.63}{{#1}}}
    \newcommand{\CommentTok}[1]{\textcolor[rgb]{0.38,0.63,0.69}{\textit{{#1}}}}
    \newcommand{\OtherTok}[1]{\textcolor[rgb]{0.00,0.44,0.13}{{#1}}}
    \newcommand{\AlertTok}[1]{\textcolor[rgb]{1.00,0.00,0.00}{\textbf{{#1}}}}
    \newcommand{\FunctionTok}[1]{\textcolor[rgb]{0.02,0.16,0.49}{{#1}}}
    \newcommand{\RegionMarkerTok}[1]{{#1}}
    \newcommand{\ErrorTok}[1]{\textcolor[rgb]{1.00,0.00,0.00}{\textbf{{#1}}}}
    \newcommand{\NormalTok}[1]{{#1}}
    
    % Additional commands for more recent versions of Pandoc
    \newcommand{\ConstantTok}[1]{\textcolor[rgb]{0.53,0.00,0.00}{{#1}}}
    \newcommand{\SpecialCharTok}[1]{\textcolor[rgb]{0.25,0.44,0.63}{{#1}}}
    \newcommand{\VerbatimStringTok}[1]{\textcolor[rgb]{0.25,0.44,0.63}{{#1}}}
    \newcommand{\SpecialStringTok}[1]{\textcolor[rgb]{0.73,0.40,0.53}{{#1}}}
    \newcommand{\ImportTok}[1]{{#1}}
    \newcommand{\DocumentationTok}[1]{\textcolor[rgb]{0.73,0.13,0.13}{\textit{{#1}}}}
    \newcommand{\AnnotationTok}[1]{\textcolor[rgb]{0.38,0.63,0.69}{\textbf{\textit{{#1}}}}}
    \newcommand{\CommentVarTok}[1]{\textcolor[rgb]{0.38,0.63,0.69}{\textbf{\textit{{#1}}}}}
    \newcommand{\VariableTok}[1]{\textcolor[rgb]{0.10,0.09,0.49}{{#1}}}
    \newcommand{\ControlFlowTok}[1]{\textcolor[rgb]{0.00,0.44,0.13}{\textbf{{#1}}}}
    \newcommand{\OperatorTok}[1]{\textcolor[rgb]{0.40,0.40,0.40}{{#1}}}
    \newcommand{\BuiltInTok}[1]{{#1}}
    \newcommand{\ExtensionTok}[1]{{#1}}
    \newcommand{\PreprocessorTok}[1]{\textcolor[rgb]{0.74,0.48,0.00}{{#1}}}
    \newcommand{\AttributeTok}[1]{\textcolor[rgb]{0.49,0.56,0.16}{{#1}}}
    \newcommand{\InformationTok}[1]{\textcolor[rgb]{0.38,0.63,0.69}{\textbf{\textit{{#1}}}}}
    \newcommand{\WarningTok}[1]{\textcolor[rgb]{0.38,0.63,0.69}{\textbf{\textit{{#1}}}}}
    
    
    % Define a nice break command that doesn't care if a line doesn't already
    % exist.
    \def\br{\hspace*{\fill} \\* }
    % Math Jax compatability definitions
    \def\gt{>}
    \def\lt{<}
    % Document parameters
    \title{DS-001 Documentacion Numpy y Pandas}
    
    
    

    % Pygments definitions
    
\makeatletter
\def\PY@reset{\let\PY@it=\relax \let\PY@bf=\relax%
    \let\PY@ul=\relax \let\PY@tc=\relax%
    \let\PY@bc=\relax \let\PY@ff=\relax}
\def\PY@tok#1{\csname PY@tok@#1\endcsname}
\def\PY@toks#1+{\ifx\relax#1\empty\else%
    \PY@tok{#1}\expandafter\PY@toks\fi}
\def\PY@do#1{\PY@bc{\PY@tc{\PY@ul{%
    \PY@it{\PY@bf{\PY@ff{#1}}}}}}}
\def\PY#1#2{\PY@reset\PY@toks#1+\relax+\PY@do{#2}}

\expandafter\def\csname PY@tok@w\endcsname{\def\PY@tc##1{\textcolor[rgb]{0.73,0.73,0.73}{##1}}}
\expandafter\def\csname PY@tok@c\endcsname{\let\PY@it=\textit\def\PY@tc##1{\textcolor[rgb]{0.25,0.50,0.50}{##1}}}
\expandafter\def\csname PY@tok@cp\endcsname{\def\PY@tc##1{\textcolor[rgb]{0.74,0.48,0.00}{##1}}}
\expandafter\def\csname PY@tok@k\endcsname{\let\PY@bf=\textbf\def\PY@tc##1{\textcolor[rgb]{0.00,0.50,0.00}{##1}}}
\expandafter\def\csname PY@tok@kp\endcsname{\def\PY@tc##1{\textcolor[rgb]{0.00,0.50,0.00}{##1}}}
\expandafter\def\csname PY@tok@kt\endcsname{\def\PY@tc##1{\textcolor[rgb]{0.69,0.00,0.25}{##1}}}
\expandafter\def\csname PY@tok@o\endcsname{\def\PY@tc##1{\textcolor[rgb]{0.40,0.40,0.40}{##1}}}
\expandafter\def\csname PY@tok@ow\endcsname{\let\PY@bf=\textbf\def\PY@tc##1{\textcolor[rgb]{0.67,0.13,1.00}{##1}}}
\expandafter\def\csname PY@tok@nb\endcsname{\def\PY@tc##1{\textcolor[rgb]{0.00,0.50,0.00}{##1}}}
\expandafter\def\csname PY@tok@nf\endcsname{\def\PY@tc##1{\textcolor[rgb]{0.00,0.00,1.00}{##1}}}
\expandafter\def\csname PY@tok@nc\endcsname{\let\PY@bf=\textbf\def\PY@tc##1{\textcolor[rgb]{0.00,0.00,1.00}{##1}}}
\expandafter\def\csname PY@tok@nn\endcsname{\let\PY@bf=\textbf\def\PY@tc##1{\textcolor[rgb]{0.00,0.00,1.00}{##1}}}
\expandafter\def\csname PY@tok@ne\endcsname{\let\PY@bf=\textbf\def\PY@tc##1{\textcolor[rgb]{0.82,0.25,0.23}{##1}}}
\expandafter\def\csname PY@tok@nv\endcsname{\def\PY@tc##1{\textcolor[rgb]{0.10,0.09,0.49}{##1}}}
\expandafter\def\csname PY@tok@no\endcsname{\def\PY@tc##1{\textcolor[rgb]{0.53,0.00,0.00}{##1}}}
\expandafter\def\csname PY@tok@nl\endcsname{\def\PY@tc##1{\textcolor[rgb]{0.63,0.63,0.00}{##1}}}
\expandafter\def\csname PY@tok@ni\endcsname{\let\PY@bf=\textbf\def\PY@tc##1{\textcolor[rgb]{0.60,0.60,0.60}{##1}}}
\expandafter\def\csname PY@tok@na\endcsname{\def\PY@tc##1{\textcolor[rgb]{0.49,0.56,0.16}{##1}}}
\expandafter\def\csname PY@tok@nt\endcsname{\let\PY@bf=\textbf\def\PY@tc##1{\textcolor[rgb]{0.00,0.50,0.00}{##1}}}
\expandafter\def\csname PY@tok@nd\endcsname{\def\PY@tc##1{\textcolor[rgb]{0.67,0.13,1.00}{##1}}}
\expandafter\def\csname PY@tok@s\endcsname{\def\PY@tc##1{\textcolor[rgb]{0.73,0.13,0.13}{##1}}}
\expandafter\def\csname PY@tok@sd\endcsname{\let\PY@it=\textit\def\PY@tc##1{\textcolor[rgb]{0.73,0.13,0.13}{##1}}}
\expandafter\def\csname PY@tok@si\endcsname{\let\PY@bf=\textbf\def\PY@tc##1{\textcolor[rgb]{0.73,0.40,0.53}{##1}}}
\expandafter\def\csname PY@tok@se\endcsname{\let\PY@bf=\textbf\def\PY@tc##1{\textcolor[rgb]{0.73,0.40,0.13}{##1}}}
\expandafter\def\csname PY@tok@sr\endcsname{\def\PY@tc##1{\textcolor[rgb]{0.73,0.40,0.53}{##1}}}
\expandafter\def\csname PY@tok@ss\endcsname{\def\PY@tc##1{\textcolor[rgb]{0.10,0.09,0.49}{##1}}}
\expandafter\def\csname PY@tok@sx\endcsname{\def\PY@tc##1{\textcolor[rgb]{0.00,0.50,0.00}{##1}}}
\expandafter\def\csname PY@tok@m\endcsname{\def\PY@tc##1{\textcolor[rgb]{0.40,0.40,0.40}{##1}}}
\expandafter\def\csname PY@tok@gh\endcsname{\let\PY@bf=\textbf\def\PY@tc##1{\textcolor[rgb]{0.00,0.00,0.50}{##1}}}
\expandafter\def\csname PY@tok@gu\endcsname{\let\PY@bf=\textbf\def\PY@tc##1{\textcolor[rgb]{0.50,0.00,0.50}{##1}}}
\expandafter\def\csname PY@tok@gd\endcsname{\def\PY@tc##1{\textcolor[rgb]{0.63,0.00,0.00}{##1}}}
\expandafter\def\csname PY@tok@gi\endcsname{\def\PY@tc##1{\textcolor[rgb]{0.00,0.63,0.00}{##1}}}
\expandafter\def\csname PY@tok@gr\endcsname{\def\PY@tc##1{\textcolor[rgb]{1.00,0.00,0.00}{##1}}}
\expandafter\def\csname PY@tok@ge\endcsname{\let\PY@it=\textit}
\expandafter\def\csname PY@tok@gs\endcsname{\let\PY@bf=\textbf}
\expandafter\def\csname PY@tok@gp\endcsname{\let\PY@bf=\textbf\def\PY@tc##1{\textcolor[rgb]{0.00,0.00,0.50}{##1}}}
\expandafter\def\csname PY@tok@go\endcsname{\def\PY@tc##1{\textcolor[rgb]{0.53,0.53,0.53}{##1}}}
\expandafter\def\csname PY@tok@gt\endcsname{\def\PY@tc##1{\textcolor[rgb]{0.00,0.27,0.87}{##1}}}
\expandafter\def\csname PY@tok@err\endcsname{\def\PY@bc##1{\setlength{\fboxsep}{0pt}\fcolorbox[rgb]{1.00,0.00,0.00}{1,1,1}{\strut ##1}}}
\expandafter\def\csname PY@tok@kc\endcsname{\let\PY@bf=\textbf\def\PY@tc##1{\textcolor[rgb]{0.00,0.50,0.00}{##1}}}
\expandafter\def\csname PY@tok@kd\endcsname{\let\PY@bf=\textbf\def\PY@tc##1{\textcolor[rgb]{0.00,0.50,0.00}{##1}}}
\expandafter\def\csname PY@tok@kn\endcsname{\let\PY@bf=\textbf\def\PY@tc##1{\textcolor[rgb]{0.00,0.50,0.00}{##1}}}
\expandafter\def\csname PY@tok@kr\endcsname{\let\PY@bf=\textbf\def\PY@tc##1{\textcolor[rgb]{0.00,0.50,0.00}{##1}}}
\expandafter\def\csname PY@tok@bp\endcsname{\def\PY@tc##1{\textcolor[rgb]{0.00,0.50,0.00}{##1}}}
\expandafter\def\csname PY@tok@fm\endcsname{\def\PY@tc##1{\textcolor[rgb]{0.00,0.00,1.00}{##1}}}
\expandafter\def\csname PY@tok@vc\endcsname{\def\PY@tc##1{\textcolor[rgb]{0.10,0.09,0.49}{##1}}}
\expandafter\def\csname PY@tok@vg\endcsname{\def\PY@tc##1{\textcolor[rgb]{0.10,0.09,0.49}{##1}}}
\expandafter\def\csname PY@tok@vi\endcsname{\def\PY@tc##1{\textcolor[rgb]{0.10,0.09,0.49}{##1}}}
\expandafter\def\csname PY@tok@vm\endcsname{\def\PY@tc##1{\textcolor[rgb]{0.10,0.09,0.49}{##1}}}
\expandafter\def\csname PY@tok@sa\endcsname{\def\PY@tc##1{\textcolor[rgb]{0.73,0.13,0.13}{##1}}}
\expandafter\def\csname PY@tok@sb\endcsname{\def\PY@tc##1{\textcolor[rgb]{0.73,0.13,0.13}{##1}}}
\expandafter\def\csname PY@tok@sc\endcsname{\def\PY@tc##1{\textcolor[rgb]{0.73,0.13,0.13}{##1}}}
\expandafter\def\csname PY@tok@dl\endcsname{\def\PY@tc##1{\textcolor[rgb]{0.73,0.13,0.13}{##1}}}
\expandafter\def\csname PY@tok@s2\endcsname{\def\PY@tc##1{\textcolor[rgb]{0.73,0.13,0.13}{##1}}}
\expandafter\def\csname PY@tok@sh\endcsname{\def\PY@tc##1{\textcolor[rgb]{0.73,0.13,0.13}{##1}}}
\expandafter\def\csname PY@tok@s1\endcsname{\def\PY@tc##1{\textcolor[rgb]{0.73,0.13,0.13}{##1}}}
\expandafter\def\csname PY@tok@mb\endcsname{\def\PY@tc##1{\textcolor[rgb]{0.40,0.40,0.40}{##1}}}
\expandafter\def\csname PY@tok@mf\endcsname{\def\PY@tc##1{\textcolor[rgb]{0.40,0.40,0.40}{##1}}}
\expandafter\def\csname PY@tok@mh\endcsname{\def\PY@tc##1{\textcolor[rgb]{0.40,0.40,0.40}{##1}}}
\expandafter\def\csname PY@tok@mi\endcsname{\def\PY@tc##1{\textcolor[rgb]{0.40,0.40,0.40}{##1}}}
\expandafter\def\csname PY@tok@il\endcsname{\def\PY@tc##1{\textcolor[rgb]{0.40,0.40,0.40}{##1}}}
\expandafter\def\csname PY@tok@mo\endcsname{\def\PY@tc##1{\textcolor[rgb]{0.40,0.40,0.40}{##1}}}
\expandafter\def\csname PY@tok@ch\endcsname{\let\PY@it=\textit\def\PY@tc##1{\textcolor[rgb]{0.25,0.50,0.50}{##1}}}
\expandafter\def\csname PY@tok@cm\endcsname{\let\PY@it=\textit\def\PY@tc##1{\textcolor[rgb]{0.25,0.50,0.50}{##1}}}
\expandafter\def\csname PY@tok@cpf\endcsname{\let\PY@it=\textit\def\PY@tc##1{\textcolor[rgb]{0.25,0.50,0.50}{##1}}}
\expandafter\def\csname PY@tok@c1\endcsname{\let\PY@it=\textit\def\PY@tc##1{\textcolor[rgb]{0.25,0.50,0.50}{##1}}}
\expandafter\def\csname PY@tok@cs\endcsname{\let\PY@it=\textit\def\PY@tc##1{\textcolor[rgb]{0.25,0.50,0.50}{##1}}}

\def\PYZbs{\char`\\}
\def\PYZus{\char`\_}
\def\PYZob{\char`\{}
\def\PYZcb{\char`\}}
\def\PYZca{\char`\^}
\def\PYZam{\char`\&}
\def\PYZlt{\char`\<}
\def\PYZgt{\char`\>}
\def\PYZsh{\char`\#}
\def\PYZpc{\char`\%}
\def\PYZdl{\char`\$}
\def\PYZhy{\char`\-}
\def\PYZsq{\char`\'}
\def\PYZdq{\char`\"}
\def\PYZti{\char`\~}
% for compatibility with earlier versions
\def\PYZat{@}
\def\PYZlb{[}
\def\PYZrb{]}
\makeatother


    % Exact colors from NB
    \definecolor{incolor}{rgb}{0.0, 0.0, 0.5}
    \definecolor{outcolor}{rgb}{0.545, 0.0, 0.0}



    
    % Prevent overflowing lines due to hard-to-break entities
    \sloppy 
    % Setup hyperref package
    \hypersetup{
      breaklinks=true,  % so long urls are correctly broken across lines
      colorlinks=true,
      urlcolor=urlcolor,
      linkcolor=linkcolor,
      citecolor=citecolor,
      }
    % Slightly bigger margins than the latex defaults
    
    \geometry{verbose,tmargin=1in,bmargin=1in,lmargin=1in,rmargin=1in}
    
    

    \begin{document}
    
    
    \maketitle
    
    

    
    \begin{Verbatim}[commandchars=\\\{\}]
{\color{incolor}In [{\color{incolor}1}]:} \PY{o}{\PYZpc{}}\PY{k}{load\PYZus{}ext} watermark
        \PY{o}{\PYZpc{}}\PY{k}{watermark}
\end{Verbatim}


    \begin{Verbatim}[commandchars=\\\{\}]
2019-05-30T20:57:56+02:00

CPython 3.6.5
IPython 6.4.0

compiler   : GCC 7.2.0
system     : Linux
release    : 5.1.5-arch1-2-ARCH
machine    : x86\_64
processor  : 
CPU cores  : 4
interpreter: 64bit

    \end{Verbatim}

    \section{INTRODUCCIÓN A DATA
SCIENCE}\label{introducciuxf3n-a-data-science}

En este apartado expondremos que es el Data Science (ciencia de datos),
que tipos de variables utiliza, estructuras y herramientas principales
(Numpy y Pandas).

    \subsection{Teoria y Definicion}\label{teoria-y-definicion}

La ciencia de datos es un campo interdisciplinario que involucra varios
métodos cientificos para el análisis de datos. Según la
\href{https://es.wikipedia.org/wiki/Ciencia_de_datos}{wikipedia} Data
Science se define como:

"Un concepto para unificar estadísticas, análisis de datos, aprendizaje
automático y sus métodos relacionados para comprender y analizar los
fenómenos reales, empleando técnicas y teorías extraídas de muchos
campos dentro del contexto de las matemáticas, la estadística, la
ciencia de la información y la informática."

Para ver la actual demanda de los diez puestos mejor pagados que
requieran un conocimiento de análisis de datos, visitar el siguiene
enlace:\\
\href{https://www.dataquest.io/blog/10-data-analytics-jobs/}{10
High-Paying Jobs That Require a Knowledge of Data Analytics}

\subsubsection{El Cientifico de Datos y su futuro
laboral.}\label{el-cientifico-de-datos-y-su-futuro-laboral.}

Un científico de datos debe seguir una serie de pasos en cualquiera de
sus proyectos: - Extraer datos, independientemente de la fuente y de su
volumen. - Limpiar los datos, para eliminar lo que pueda sesgar los
resultados. - Procesar los datos usando métodos estadísticos como
inferencia estadística, modelos de regresión, pruebas de hipótesis, etc.
- Diseñar experimentos adicionales en caso de ser necesario. - Crear
visualizaciones graficas de los datos relevantes de la investigación.23

Por norma general las fases distinguidas son:\\
\textbf{\emph{1. Definición de objetivos:}} Define los problemas a
solucionar, se soluciona normalmente con el cliente y los técnicos,
estudiando el objetivo a alcanzar. Para los data scientist, un buen
objetivo tiene que seguir la regla S.M.A.R.T(specific, measurable,
achievable, relevant, time-bound)

\textbf{\emph{2. Obtención de datos:}} Los datos se obtienen de
cualquier forma que podamos imaginar como desarrolladores, desde bases
de datos, hasta archivos csv, excel etc... La obtención de datos es una
de las fases más importantes en el desarrollo del proyecto, ya que
cuantas mas completos y extensos sean los registros, más preciso será el
analisis

    \subsection{Tipos de datos. Variables y
Estructuras}\label{tipos-de-datos.-variables-y-estructuras}

Los datos puden dividirse en los siguientes tipos de variabes: -
\emph{continuas:} edad,altura,colores RGB etc. - \emph{ordinales:}
rating, niveles educativos etc. - \emph{categóricas:} valores booleanos,
días de la semana etc.

A su vez, pueden categorizarse según su estructura: -
\emph{Estructurados \textless{}10\%:}Son datos que se relacionan entre
sí y comparten informacón como el catalogo de biblioteca, bases de datos
sql. - \emph{Semiestructurados \textless{}10\%: no tienen estructura,
pero es facil asignarle una estructura mediante la lógica.} xml, json,
csv. - \emph{No estructurados:} emails, fotos, pdf. Hoy en día más del
80\% de los datos son no estructurados, por lo que perdemos mucha
información al dificultarnos a nosotros mismos el análisis.

    \subsection{Numpy}\label{numpy}

Numpy es la piedra angular de la computación científica en Python. Nos
permite trabajar con array 'n' dimensionales, los cuales nos
proporcionan ventajas frente a las listas de Python.

Numpy a bajo nivel esta compilado en C, y al trabajar con arrays (la
disposición en las celdas de memoria frente a las listas) es una
herramienta muy potente para trabajar en Data Science con Python.

Enlace a la pagina oficial: \url{https://www.numpy.org/}

\subsubsection{Preparacion del Entorno.}\label{preparacion-del-entorno.}

Para poder trabajar con Numpy (y más librerías detalladas en los
siguietes documentos) necesitamos activar un entorno desde la terminal.

En Linux:

\begin{Shaded}
\begin{Highlighting}[]
\BuiltInTok{source}\NormalTok{ activate data}
\end{Highlighting}
\end{Shaded}

En Windows:

\begin{verbatim}
activate data
\end{verbatim}

En ambos casos 'data' sera el nombre del entorno. Ahora procederemos a
instalar en el entorno \textbf{Numpy} mediante \textbf{conda}

\begin{Shaded}
\begin{Highlighting}[]
\ExtensionTok{conda}\NormalTok{ install numpy}
\end{Highlighting}
\end{Shaded}

Nos preguntará si queremos instalarlo, marcamos 'y' y pulsamos 'enter'.

\subsubsection{Creacion de numpy arrays}\label{creacion-de-numpy-arrays}

    \begin{Verbatim}[commandchars=\\\{\}]
{\color{incolor}In [{\color{incolor}2}]:} \PY{k+kn}{import} \PY{n+nn}{sys}
        \PY{k+kn}{import} \PY{n+nn}{numpy} \PY{k}{as} \PY{n+nn}{np}
\end{Verbatim}


    Aqui explicaremos mediante markdown el significado de las variables y
para que utilizamos la herramientas.

\paragraph{Vectores.}\label{vectores.}

    \begin{Verbatim}[commandchars=\\\{\}]
{\color{incolor}In [{\color{incolor}3}]:} \PY{c+c1}{\PYZsh{}Instanciacion de un array de 1 dimension.}
        \PY{n}{array\PYZus{}1d} \PY{o}{=} \PY{n}{np}\PY{o}{.}\PY{n}{array}\PY{p}{(}\PY{p}{[}\PY{l+m+mi}{4}\PY{p}{,}\PY{l+m+mi}{5}\PY{p}{,} \PY{l+m+mi}{3}\PY{p}{]}\PY{p}{)}
        \PY{n+nb}{type}\PY{p}{(}\PY{n}{array\PYZus{}1d}\PY{p}{)}
\end{Verbatim}


\begin{Verbatim}[commandchars=\\\{\}]
{\color{outcolor}Out[{\color{outcolor}3}]:} numpy.ndarray
\end{Verbatim}
            
    \begin{Verbatim}[commandchars=\\\{\}]
{\color{incolor}In [{\color{incolor}4}]:} \PY{c+c1}{\PYZsh{}np.ones genera un vector de longitud 3 inicializado con todos}
        \PY{c+c1}{\PYZsh{}los valores a 1}
        \PY{n+nb}{print}\PY{p}{(}\PY{l+s+s2}{\PYZdq{}}\PY{l+s+s2}{np.ones}\PY{l+s+se}{\PYZbs{}n}\PY{l+s+s2}{\PYZdq{}}\PY{p}{,}\PY{n}{np}\PY{o}{.}\PY{n}{ones}\PY{p}{(}\PY{l+m+mi}{3}\PY{p}{)}\PY{p}{)}
\end{Verbatim}


    \begin{Verbatim}[commandchars=\\\{\}]
np.ones
 [1. 1. 1.]

    \end{Verbatim}

    \paragraph{Matrices.}\label{matrices.}

    \begin{Verbatim}[commandchars=\\\{\}]
{\color{incolor}In [{\color{incolor}5}]:} \PY{c+c1}{\PYZsh{}Instanciacion de una matriz}
        \PY{n}{matriz} \PY{o}{=} \PY{n}{np}\PY{o}{.}\PY{n}{array}\PY{p}{(}\PY{p}{[}
            \PY{p}{[} \PY{l+m+mi}{1}\PY{p}{,}\PY{l+m+mi}{2}\PY{p}{,} \PY{l+m+mi}{1} \PY{p}{]}\PY{p}{,}
            \PY{p}{[}\PY{l+m+mi}{5}\PY{p}{,} \PY{l+m+mi}{43}\PY{p}{,} \PY{l+m+mi}{5}\PY{p}{]}
        \PY{p}{]}\PY{p}{)}
        
        \PY{n}{matriz}
\end{Verbatim}


\begin{Verbatim}[commandchars=\\\{\}]
{\color{outcolor}Out[{\color{outcolor}5}]:} array([[ 1,  2,  1],
               [ 5, 43,  5]])
\end{Verbatim}
            
    \begin{Verbatim}[commandchars=\\\{\}]
{\color{incolor}In [{\color{incolor}6}]:} \PY{c+c1}{\PYZsh{}np.eye genera una matriz identidad de 3x3}
        \PY{n+nb}{print}\PY{p}{(}\PY{l+s+s2}{\PYZdq{}}\PY{l+s+s2}{np.eye}\PY{l+s+se}{\PYZbs{}n}\PY{l+s+s2}{\PYZdq{}}\PY{p}{,}\PY{n}{np}\PY{o}{.}\PY{n}{eye}\PY{p}{(}\PY{l+m+mi}{3}\PY{p}{)}\PY{p}{)}
\end{Verbatim}


    \begin{Verbatim}[commandchars=\\\{\}]
np.eye
 [[1. 0. 0.]
 [0. 1. 0.]
 [0. 0. 1.]]

    \end{Verbatim}

    \begin{Verbatim}[commandchars=\\\{\}]
{\color{incolor}In [{\color{incolor}7}]:} \PY{c+c1}{\PYZsh{}np.zeros genera una matriz con todos sus valores a 0}
        \PY{n+nb}{print}\PY{p}{(}\PY{l+s+s2}{\PYZdq{}}\PY{l+s+s2}{np.zeros}\PY{l+s+se}{\PYZbs{}n}\PY{l+s+s2}{\PYZdq{}}\PY{p}{,}\PY{n}{np}\PY{o}{.}\PY{n}{zeros}\PY{p}{(}\PY{p}{(}\PY{l+m+mi}{3}\PY{p}{,}\PY{l+m+mi}{2}\PY{p}{)}\PY{p}{)}\PY{p}{)}
\end{Verbatim}


    \begin{Verbatim}[commandchars=\\\{\}]
np.zeros
 [[0. 0.]
 [0. 0.]
 [0. 0.]]

    \end{Verbatim}

    \begin{Verbatim}[commandchars=\\\{\}]
{\color{incolor}In [{\color{incolor}8}]:} \PY{c+c1}{\PYZsh{}np.random produce un array con valores aleatorios entre el intervalo [0,1]}
        \PY{n}{np}\PY{o}{.}\PY{n}{random}\PY{o}{.}\PY{n}{random}\PY{p}{(}\PY{p}{(}\PY{l+m+mi}{2}\PY{p}{,}\PY{l+m+mi}{3}\PY{p}{)}\PY{p}{)}
\end{Verbatim}


\begin{Verbatim}[commandchars=\\\{\}]
{\color{outcolor}Out[{\color{outcolor}8}]:} array([[0.14252823, 0.84562171, 0.23374288],
               [0.64021973, 0.14124967, 0.93168794]])
\end{Verbatim}
            
    \paragraph{Flujo de lectura y volcado en
array.}\label{flujo-de-lectura-y-volcado-en-array.}

    \begin{Verbatim}[commandchars=\\\{\}]
{\color{incolor}In [{\color{incolor}9}]:} \PY{c+c1}{\PYZsh{}se puede acceder a un documento de texto y volcarlo}
        \PY{c+c1}{\PYZsh{}a un numpy array}
        \PY{n}{np\PYZus{}text} \PY{o}{=} \PY{n}{np}\PY{o}{.}\PY{n}{genfromtxt}\PY{p}{(}\PY{l+s+s2}{\PYZdq{}}\PY{l+s+s2}{np\PYZus{}text.txt}\PY{l+s+s2}{\PYZdq{}}\PY{p}{,} \PY{n}{delimiter}\PY{o}{=}\PY{l+s+s2}{\PYZdq{}}\PY{l+s+s2}{,}\PY{l+s+s2}{\PYZdq{}}\PY{p}{)}
        \PY{n}{np\PYZus{}text}
\end{Verbatim}


\begin{Verbatim}[commandchars=\\\{\}]
{\color{outcolor}Out[{\color{outcolor}9}]:} array([[ 1.,  2.,  3.],
               [43.,  2.,  3.],
               [34.,  1.,  1.],
               [ 0.,  1.,  1.]])
\end{Verbatim}
            
    \paragraph{Seleccion por secciones y por indices (slicing e
indexing)}\label{seleccion-por-secciones-y-por-indices-slicing-e-indexing}

    \begin{Verbatim}[commandchars=\\\{\}]
{\color{incolor}In [{\color{incolor}10}]:} \PY{c+c1}{\PYZsh{}instanciamos matriz de ejemplo}
         \PY{n}{matriz\PYZus{}34} \PY{o}{=} \PY{n}{np}\PY{o}{.}\PY{n}{array}\PY{p}{(}\PY{p}{[}\PY{p}{[}\PY{l+m+mi}{1}\PY{p}{,}\PY{l+m+mi}{2}\PY{p}{,}\PY{l+m+mi}{3}\PY{p}{,}\PY{l+m+mi}{4}\PY{p}{]}\PY{p}{,} \PY{p}{[}\PY{l+m+mi}{5}\PY{p}{,}\PY{l+m+mi}{6}\PY{p}{,}\PY{l+m+mi}{7}\PY{p}{,}\PY{l+m+mi}{8}\PY{p}{]}\PY{p}{,} \PY{p}{[}\PY{l+m+mi}{9}\PY{p}{,}\PY{l+m+mi}{10}\PY{p}{,}\PY{l+m+mi}{11}\PY{p}{,}\PY{l+m+mi}{12}\PY{p}{]}\PY{p}{]}\PY{p}{)}
         \PY{n}{matriz\PYZus{}34}
\end{Verbatim}


\begin{Verbatim}[commandchars=\\\{\}]
{\color{outcolor}Out[{\color{outcolor}10}]:} array([[ 1,  2,  3,  4],
                [ 5,  6,  7,  8],
                [ 9, 10, 11, 12]])
\end{Verbatim}
            
    \begin{Verbatim}[commandchars=\\\{\}]
{\color{incolor}In [{\color{incolor}11}]:} \PY{c+c1}{\PYZsh{}obtenemos su primera fila como si de una lista se tratase}
         \PY{n}{matriz\PYZus{}34}\PY{p}{[}\PY{l+m+mi}{0}\PY{p}{]}
\end{Verbatim}


\begin{Verbatim}[commandchars=\\\{\}]
{\color{outcolor}Out[{\color{outcolor}11}]:} array([1, 2, 3, 4])
\end{Verbatim}
            
    \begin{Verbatim}[commandchars=\\\{\}]
{\color{incolor}In [{\color{incolor}12}]:} \PY{c+c1}{\PYZsh{}seleccionamos ahora hasta la fila 2 (fila 1 y fila 2)}
         \PY{n}{matriz\PYZus{}34}\PY{p}{[}\PY{p}{:}\PY{l+m+mi}{2}\PY{p}{]}
\end{Verbatim}


\begin{Verbatim}[commandchars=\\\{\}]
{\color{outcolor}Out[{\color{outcolor}12}]:} array([[1, 2, 3, 4],
                [5, 6, 7, 8]])
\end{Verbatim}
            
    \begin{Verbatim}[commandchars=\\\{\}]
{\color{incolor}In [{\color{incolor}13}]:} \PY{c+c1}{\PYZsh{}podemos tambien seleccionar el segundo elemento de cada fila}
         \PY{c+c1}{\PYZsh{}es decir, la segunda columna}
         \PY{n}{matriz\PYZus{}34}\PY{p}{[}\PY{p}{:}\PY{p}{,}\PY{l+m+mi}{1}\PY{p}{]}
\end{Verbatim}


\begin{Verbatim}[commandchars=\\\{\}]
{\color{outcolor}Out[{\color{outcolor}13}]:} array([ 2,  6, 10])
\end{Verbatim}
            
    \begin{Verbatim}[commandchars=\\\{\}]
{\color{incolor}In [{\color{incolor}14}]:} \PY{c+c1}{\PYZsh{}al crear secciones solo obtenemos un puntero o referencia}
         \PY{c+c1}{\PYZsh{}al mismo array, no instanciamos nuevos objetos.}
         \PY{n}{seccion} \PY{o}{=} \PY{n}{matriz\PYZus{}34}\PY{p}{[}\PY{p}{:}\PY{l+m+mi}{2}\PY{p}{,}\PY{p}{:}\PY{p}{]}
         \PY{n+nb}{print}\PY{p}{(}\PY{n}{matriz\PYZus{}34}\PY{p}{[}\PY{l+m+mi}{0}\PY{p}{,}\PY{l+m+mi}{1}\PY{p}{]}\PY{p}{)}
         \PY{n}{seccion}\PY{p}{[}\PY{l+m+mi}{0}\PY{p}{,} \PY{l+m+mi}{0}\PY{p}{]} \PY{o}{=} \PY{l+m+mi}{0}
         \PY{n}{matriz\PYZus{}34}
\end{Verbatim}


    \begin{Verbatim}[commandchars=\\\{\}]
2

    \end{Verbatim}

\begin{Verbatim}[commandchars=\\\{\}]
{\color{outcolor}Out[{\color{outcolor}14}]:} array([[ 0,  2,  3,  4],
                [ 5,  6,  7,  8],
                [ 9, 10, 11, 12]])
\end{Verbatim}
            
    \paragraph{Filtrado}\label{filtrado}

    \begin{Verbatim}[commandchars=\\\{\}]
{\color{incolor}In [{\color{incolor}15}]:} \PY{n}{matriz\PYZus{}32} \PY{o}{=} \PY{n}{np}\PY{o}{.}\PY{n}{array}\PY{p}{(}\PY{p}{[}\PY{p}{[}\PY{l+m+mi}{1}\PY{p}{,} \PY{l+m+mi}{4}\PY{p}{]}\PY{p}{,} \PY{p}{[}\PY{l+m+mi}{2}\PY{p}{,} \PY{l+m+mi}{4}\PY{p}{]}\PY{p}{,} \PY{p}{[}\PY{l+m+mi}{5}\PY{p}{,} \PY{l+m+mi}{0}\PY{p}{]}\PY{p}{]}\PY{p}{)}
         \PY{n}{matriz\PYZus{}32}
\end{Verbatim}


\begin{Verbatim}[commandchars=\\\{\}]
{\color{outcolor}Out[{\color{outcolor}15}]:} array([[1, 4],
                [2, 4],
                [5, 0]])
\end{Verbatim}
            
    \begin{Verbatim}[commandchars=\\\{\}]
{\color{incolor}In [{\color{incolor}16}]:} \PY{c+c1}{\PYZsh{}comprobamos aquellos elementos que sean mayores o iguales que 2}
         \PY{n}{indice\PYZus{}fitrado} \PY{o}{=} \PY{p}{(}\PY{n}{matriz\PYZus{}32} \PY{o}{\PYZgt{}}\PY{o}{=} \PY{l+m+mi}{2}\PY{p}{)}
         \PY{n}{indice\PYZus{}fitrado}
\end{Verbatim}


\begin{Verbatim}[commandchars=\\\{\}]
{\color{outcolor}Out[{\color{outcolor}16}]:} array([[False,  True],
                [ True,  True],
                [ True, False]])
\end{Verbatim}
            
    \begin{Verbatim}[commandchars=\\\{\}]
{\color{incolor}In [{\color{incolor}17}]:} \PY{n}{matriz\PYZus{}32}\PY{p}{[}\PY{n}{indice\PYZus{}fitrado}\PY{p}{]}
\end{Verbatim}


\begin{Verbatim}[commandchars=\\\{\}]
{\color{outcolor}Out[{\color{outcolor}17}]:} array([4, 2, 4, 5])
\end{Verbatim}
            
    \paragraph{Aritmetica con numpy
arrays}\label{aritmetica-con-numpy-arrays}

    \begin{Verbatim}[commandchars=\\\{\}]
{\color{incolor}In [{\color{incolor}18}]:} \PY{n}{array1} \PY{o}{=} \PY{n}{np}\PY{o}{.}\PY{n}{array}\PY{p}{(}\PY{p}{[}\PY{p}{[}\PY{l+m+mi}{2}\PY{p}{,}\PY{l+m+mi}{3}\PY{p}{]}\PY{p}{,}\PY{p}{[}\PY{l+m+mi}{0}\PY{p}{,}\PY{l+m+mi}{1}\PY{p}{]}\PY{p}{]}\PY{p}{)}
         \PY{n}{array2} \PY{o}{=} \PY{n}{np}\PY{o}{.}\PY{n}{array}\PY{p}{(}\PY{p}{[}\PY{p}{[}\PY{l+m+mi}{23}\PY{p}{,}\PY{l+m+mi}{6}\PY{p}{]}\PY{p}{,}\PY{p}{[}\PY{l+m+mi}{0}\PY{p}{,}\PY{l+m+mi}{42}\PY{p}{]}\PY{p}{]}\PY{p}{)}
         
         \PY{n+nb}{print}\PY{p}{(}\PY{n}{array1}\PY{p}{)}
         \PY{n+nb}{print}\PY{p}{(}\PY{n}{array2}\PY{p}{)}
\end{Verbatim}


    \begin{Verbatim}[commandchars=\\\{\}]
[[2 3]
 [0 1]]
[[23  6]
 [ 0 42]]

    \end{Verbatim}

    \begin{Verbatim}[commandchars=\\\{\}]
{\color{incolor}In [{\color{incolor}19}]:} \PY{n}{array1} \PY{o}{+} \PY{n}{array2}
\end{Verbatim}


\begin{Verbatim}[commandchars=\\\{\}]
{\color{outcolor}Out[{\color{outcolor}19}]:} array([[25,  9],
                [ 0, 43]])
\end{Verbatim}
            
    \begin{Verbatim}[commandchars=\\\{\}]
{\color{incolor}In [{\color{incolor}20}]:} \PY{n}{array1} \PY{o}{*} \PY{n}{array2}
\end{Verbatim}


\begin{Verbatim}[commandchars=\\\{\}]
{\color{outcolor}Out[{\color{outcolor}20}]:} array([[46, 18],
                [ 0, 42]])
\end{Verbatim}
            
    Desde python 3.5, podemos usar el simbolo \texttt{@} para indicar una
multiplicación de matrices (para versiones anteriores se usa la funcion
\texttt{dot}

    \begin{Verbatim}[commandchars=\\\{\}]
{\color{incolor}In [{\color{incolor}21}]:} \PY{n}{array1} \PY{o}{@} \PY{n}{array2}
\end{Verbatim}


\begin{Verbatim}[commandchars=\\\{\}]
{\color{outcolor}Out[{\color{outcolor}21}]:} array([[ 46, 138],
                [  0,  42]])
\end{Verbatim}
            
    este producto equivale a

    \begin{Verbatim}[commandchars=\\\{\}]
{\color{incolor}In [{\color{incolor}22}]:} \PY{n}{array1}\PY{o}{.}\PY{n}{dot}\PY{p}{(}\PY{n}{array2}\PY{p}{)}
\end{Verbatim}


\begin{Verbatim}[commandchars=\\\{\}]
{\color{outcolor}Out[{\color{outcolor}22}]:} array([[ 46, 138],
                [  0,  42]])
\end{Verbatim}
            
    \paragraph{Ventajas de np.array vs
lists.}\label{ventajas-de-np.array-vs-lists.}

    \begin{Verbatim}[commandchars=\\\{\}]
{\color{incolor}In [{\color{incolor}23}]:} \PY{n}{lista\PYZus{}2d} \PY{o}{=} \PY{p}{[}\PY{p}{[}\PY{l+m+mi}{1222}\PY{p}{,}\PY{l+m+mi}{2222}\PY{p}{,}\PY{l+m+mi}{2223}\PY{p}{]}\PY{p}{,} \PY{p}{[}\PY{l+m+mi}{5}\PY{p}{,}\PY{l+m+mi}{23}\PY{p}{,}\PY{l+m+mi}{40004}\PY{p}{]}\PY{p}{]}
         
         \PY{n}{array\PYZus{}2d} \PY{o}{=} \PY{n}{np}\PY{o}{.}\PY{n}{array}\PY{p}{(}\PY{p}{[}\PY{p}{[}\PY{l+m+mi}{1222}\PY{p}{,}\PY{l+m+mi}{2222}\PY{p}{,}\PY{l+m+mi}{2223}\PY{p}{]}\PY{p}{,} \PY{p}{[}\PY{l+m+mi}{5}\PY{p}{,}\PY{l+m+mi}{23}\PY{p}{,}\PY{l+m+mi}{40004}\PY{p}{]}\PY{p}{]}\PY{p}{)}
         
         \PY{n+nb}{print}\PY{p}{(}\PY{l+s+s2}{\PYZdq{}}\PY{l+s+s2}{Tamaño de la lista en memoria: }\PY{l+s+si}{\PYZob{}\PYZcb{}}\PY{l+s+s2}{ bytes}\PY{l+s+s2}{\PYZdq{}}\PY{o}{.}\PY{n}{format}\PY{p}{(}\PY{n}{sys}\PY{o}{.}\PY{n}{getsizeof}\PY{p}{(}\PY{n}{lista\PYZus{}2d}\PY{p}{)}\PY{p}{)}\PY{p}{)}
         \PY{n+nb}{print}\PY{p}{(}\PY{l+s+s2}{\PYZdq{}}\PY{l+s+s2}{Tamaño del numpy array en memoria: }\PY{l+s+si}{\PYZob{}\PYZcb{}}\PY{l+s+s2}{ bytes}\PY{l+s+s2}{\PYZdq{}}\PY{o}{.}\PY{n}{format}\PY{p}{(}\PY{n}{sys}\PY{o}{.}\PY{n}{getsizeof}\PY{p}{(}\PY{n}{array\PYZus{}2d}\PY{p}{)}\PY{p}{)}\PY{p}{)}
\end{Verbatim}


    \begin{Verbatim}[commandchars=\\\{\}]
Tamaño de la lista en memoria: 80 bytes
Tamaño del numpy array en memoria: 160 bytes

    \end{Verbatim}

    \begin{Verbatim}[commandchars=\\\{\}]
{\color{incolor}In [{\color{incolor}24}]:} \PY{n}{big\PYZus{}list} \PY{o}{=} \PY{n+nb}{list}\PY{p}{(}\PY{n+nb}{range}\PY{p}{(}\PY{l+m+mi}{10000}\PY{p}{)}\PY{p}{)}
         \PY{n}{big\PYZus{}array} \PY{o}{=} \PY{n}{np}\PY{o}{.}\PY{n}{array}\PY{p}{(}\PY{n+nb}{range}\PY{p}{(}\PY{l+m+mi}{100000}\PY{p}{)}\PY{p}{)}
         
         \PY{n+nb}{print}\PY{p}{(}\PY{l+s+s2}{\PYZdq{}}\PY{l+s+s2}{Tamaño de la lista en memoria: }\PY{l+s+si}{\PYZob{}\PYZcb{}}\PY{l+s+s2}{ bytes}\PY{l+s+s2}{\PYZdq{}}\PY{o}{.}\PY{n}{format}\PY{p}{(}\PY{n}{sys}\PY{o}{.}\PY{n}{getsizeof}\PY{p}{(}\PY{n}{big\PYZus{}list}\PY{p}{)}\PY{p}{)}\PY{p}{)}
         \PY{n+nb}{print}\PY{p}{(}\PY{l+s+s2}{\PYZdq{}}\PY{l+s+s2}{Tamaño del numpy array en memoria: }\PY{l+s+si}{\PYZob{}\PYZcb{}}\PY{l+s+s2}{ bytes}\PY{l+s+s2}{\PYZdq{}}\PY{o}{.}\PY{n}{format}\PY{p}{(}\PY{n}{sys}\PY{o}{.}\PY{n}{getsizeof}\PY{p}{(}\PY{n}{big\PYZus{}array}\PY{p}{)}\PY{p}{)}\PY{p}{)}
\end{Verbatim}


    \begin{Verbatim}[commandchars=\\\{\}]
Tamaño de la lista en memoria: 90112 bytes
Tamaño del numpy array en memoria: 800096 bytes

    \end{Verbatim}

    \subsection{Pandas.}\label{pandas.}

\href{https://pandas.pydata.org/}{Pandas} es una librería escrita como
extensión de numpy para manipulación y análisis de datos para el
lenguaje de programación Python.

\subsubsection{Preparación del
Entorno}\label{preparaciuxf3n-del-entorno}

Para su instalación en el entorno pueden utilizarse los siguientes
comandos:

\begin{itemize}
\item
  Via Conda:

\begin{Shaded}
\begin{Highlighting}[]
\ExtensionTok{conda}\NormalTok{ install pandas}
\end{Highlighting}
\end{Shaded}
\item
  Via conda forge:

\begin{Shaded}
\begin{Highlighting}[]
\ExtensionTok{conda}\NormalTok{ install -c conda-forge pandas}
\end{Highlighting}
\end{Shaded}
\item
  Vida PyPI:

\begin{Shaded}
\begin{Highlighting}[]
\ExtensionTok{python3}\NormalTok{ -m pip install --upgrade pandas}
\end{Highlighting}
\end{Shaded}

  \textbf{Las características de la biblioteca son:}
\item
  \textbf{El tipo de datos son DataFrame} para manipulación de datos con
  indexación integrada. Tiene herramientas para leer y escribir datos
  entre estructuras de dato en memoria y formatos de archivos variados\\
\item
  Permite la alineación de dato y manejo integrado de datos fallantes,
  la reestructurción y segmentación de conjuntos de datos, la
  segmentación vertical basada en etiquetas, indexación elegante, y
  segmentación horizontal de grandes conjuntos de datos, la inserción y
  eliminación de columnas en estructuras de datos.\\
\item
  Puedes realizar cadenas de operaciones, dividir, aplicar y combinar
  sobre conjuntos de datos, la mezcla y unión de datos.
\item
  Permite realizar indexación jerárquica de ejes para trabajar con datos
  de altas dimensiones en estructuras de datos de menor dimensión, la
  funcionalidad de series de tiempo: generación de rangos de fechas y
  conversión de frecuencias, desplazamiento de ventanas estadísticas y
  de regresiones lineales, desplazamiento de fechas y retrasos.
\end{itemize}

fuente:
https://www.master-data-scientist.com/pandas-herramienta-data-science/

    \subsubsection{Creación de un
DataFrame}\label{creaciuxf3n-de-un-dataframe}

    \begin{Verbatim}[commandchars=\\\{\}]
{\color{incolor}In [{\color{incolor}25}]:} \PY{k+kn}{import} \PY{n+nn}{pandas} \PY{k}{as} \PY{n+nn}{pd}
\end{Verbatim}


    \begin{Verbatim}[commandchars=\\\{\}]
{\color{incolor}In [{\color{incolor}26}]:} \PY{n}{pd}\PY{o}{.}\PY{n}{\PYZus{}\PYZus{}version\PYZus{}\PYZus{}}
\end{Verbatim}


\begin{Verbatim}[commandchars=\\\{\}]
{\color{outcolor}Out[{\color{outcolor}26}]:} '0.24.2'
\end{Verbatim}
            
    \begin{Verbatim}[commandchars=\\\{\}]
{\color{incolor}In [{\color{incolor}27}]:} \PY{c+c1}{\PYZsh{}instanciacion de un DataFrame.}
         \PY{n}{rick\PYZus{}and\PYZus{}morty} \PY{o}{=} \PY{n}{pd}\PY{o}{.}\PY{n}{DataFrame}\PY{p}{(}
         \PY{p}{\PYZob{}}
             \PY{l+s+s2}{\PYZdq{}}\PY{l+s+s2}{nombre}\PY{l+s+s2}{\PYZdq{}}\PY{p}{:}\PY{p}{[}\PY{l+s+s2}{\PYZdq{}}\PY{l+s+s2}{Rick}\PY{l+s+s2}{\PYZdq{}}\PY{p}{,}\PY{l+s+s2}{\PYZdq{}}\PY{l+s+s2}{Morty}\PY{l+s+s2}{\PYZdq{}}\PY{p}{]}\PY{p}{,}
             \PY{l+s+s2}{\PYZdq{}}\PY{l+s+s2}{apellidos}\PY{l+s+s2}{\PYZdq{}}\PY{p}{:}\PY{p}{[}\PY{l+s+s2}{\PYZdq{}}\PY{l+s+s2}{Sanchez}\PY{l+s+s2}{\PYZdq{}}\PY{p}{,}\PY{l+s+s2}{\PYZdq{}}\PY{l+s+s2}{Smith}\PY{l+s+s2}{\PYZdq{}}\PY{p}{]}\PY{p}{,}
             \PY{l+s+s2}{\PYZdq{}}\PY{l+s+s2}{edad}\PY{l+s+s2}{\PYZdq{}}\PY{p}{:}\PY{p}{[}\PY{l+m+mi}{60}\PY{p}{,}\PY{l+m+mi}{14}\PY{p}{]}
         \PY{p}{\PYZcb{}}\PY{p}{)}
         \PY{n}{rick\PYZus{}and\PYZus{}morty}
\end{Verbatim}


\begin{Verbatim}[commandchars=\\\{\}]
{\color{outcolor}Out[{\color{outcolor}27}]:}   nombre apellidos  edad
         0   Rick   Sanchez    60
         1  Morty     Smith    14
\end{Verbatim}
            
    \begin{Verbatim}[commandchars=\\\{\}]
{\color{incolor}In [{\color{incolor}28}]:}  \PY{c+c1}{\PYZsh{}Observamos el tipo de objeto de la variable rick\PYZus{}and\PYZus{}morty}
         \PY{n+nb}{type}\PY{p}{(}\PY{n}{rick\PYZus{}and\PYZus{}morty}\PY{p}{)}
\end{Verbatim}


\begin{Verbatim}[commandchars=\\\{\}]
{\color{outcolor}Out[{\color{outcolor}28}]:} pandas.core.frame.DataFrame
\end{Verbatim}
            
    También podemos instanciar un dataframe pasandole listas y especificando
en el segundo parámetro el nombre de las columnas

    \begin{Verbatim}[commandchars=\\\{\}]
{\color{incolor}In [{\color{incolor}29}]:} \PY{n}{rick\PYZus{}and\PYZus{}morty} \PY{o}{=} \PY{n}{pd}\PY{o}{.}\PY{n}{DataFrame}\PY{p}{(}
             \PY{p}{[}
                 \PY{p}{[}\PY{l+s+s2}{\PYZdq{}}\PY{l+s+s2}{Rick}\PY{l+s+s2}{\PYZdq{}}\PY{p}{,}\PY{l+s+s2}{\PYZdq{}}\PY{l+s+s2}{Sanchez}\PY{l+s+s2}{\PYZdq{}}\PY{p}{,}\PY{l+m+mi}{60}\PY{p}{]}\PY{p}{,}
                 \PY{p}{[}\PY{l+s+s2}{\PYZdq{}}\PY{l+s+s2}{Morty}\PY{l+s+s2}{\PYZdq{}}\PY{p}{,}\PY{l+s+s2}{\PYZdq{}}\PY{l+s+s2}{Smith}\PY{l+s+s2}{\PYZdq{}}\PY{p}{,}\PY{l+m+mi}{14}\PY{p}{]}
             \PY{p}{]}\PY{p}{,}\PY{n}{columns} \PY{o}{=} \PY{p}{[}\PY{l+s+s2}{\PYZdq{}}\PY{l+s+s2}{nombre}\PY{l+s+s2}{\PYZdq{}}\PY{p}{,}\PY{l+s+s2}{\PYZdq{}}\PY{l+s+s2}{apellidos}\PY{l+s+s2}{\PYZdq{}}\PY{p}{,}\PY{l+s+s2}{\PYZdq{}}\PY{l+s+s2}{edad}\PY{l+s+s2}{\PYZdq{}}\PY{p}{]}
         \PY{p}{)}
         
         \PY{n}{rick\PYZus{}and\PYZus{}morty}
\end{Verbatim}


\begin{Verbatim}[commandchars=\\\{\}]
{\color{outcolor}Out[{\color{outcolor}29}]:}   nombre apellidos  edad
         0   Rick   Sanchez    60
         1  Morty     Smith    14
\end{Verbatim}
            
    \subsubsection{Lectura de Ficheros CSV}\label{lectura-de-ficheros-csv}

    Lo mas habitual al trabajar con dataframes de pandas es cargar los datos
del mismo de un archivo \textbf{csv}. Por convención, cuando trabajamos
con un dataframe "generico" se le suele nombrar \textbf{df}

    \begin{Verbatim}[commandchars=\\\{\}]
{\color{incolor}In [{\color{incolor}30}]:} \PY{c+c1}{\PYZsh{}Flujo input para leer csv}
         \PY{n}{df} \PY{o}{=} \PY{n}{pd}\PY{o}{.}\PY{n}{read\PYZus{}csv}\PY{p}{(}\PY{l+s+s2}{\PYZdq{}}\PY{l+s+s2}{primary\PYZus{}results.csv}\PY{l+s+s2}{\PYZdq{}}\PY{p}{)}
         \PY{n}{df}
         
         \PY{c+c1}{\PYZsh{}Flujo output para escribir csv}
         \PY{n}{df}\PY{o}{.}\PY{n}{to\PYZus{}csv}\PY{p}{(}\PY{l+s+s2}{\PYZdq{}}\PY{l+s+s2}{test.csv}\PY{l+s+s2}{\PYZdq{}}\PY{p}{)}
\end{Verbatim}


    \subsubsection{Lectura desde Clipboard}\label{lectura-desde-clipboard}

Si seleccionamos y copiamos un fragemnto de Dataframe se podra mostrar
posteriormente gracias al metodo \textbf{read\_clipboard}

    \begin{Verbatim}[commandchars=\\\{\}]
{\color{incolor}In [{\color{incolor}31}]:} \PY{n}{df} \PY{o}{=} \PY{n}{pd}\PY{o}{.}\PY{n}{read\PYZus{}clipboard}\PY{p}{(}\PY{p}{)}
         \PY{n}{df}
\end{Verbatim}


\begin{Verbatim}[commandchars=\\\{\}]
{\color{outcolor}Out[{\color{outcolor}31}]:} Empty DataFrame
         Columns: [http://localhost:8888/login?next=\%2Ftree]
         Index: []
\end{Verbatim}
            
    \subsubsection{Exploración de
DataFrames}\label{exploraciuxf3n-de-dataframes}

Para la demostración de los métodos de exploración de Dataframes,
utilizaremos el dataframe de los resultado de las votaciones primarias
en Estados Unidos.

    \begin{Verbatim}[commandchars=\\\{\}]
{\color{incolor}In [{\color{incolor}32}]:} \PY{n}{df} \PY{o}{=} \PY{n}{pd}\PY{o}{.}\PY{n}{read\PYZus{}csv}\PY{p}{(}\PY{l+s+s2}{\PYZdq{}}\PY{l+s+s2}{primary\PYZus{}results.csv}\PY{l+s+s2}{\PYZdq{}}\PY{p}{)}
\end{Verbatim}


    Para obtener el numero total de registos y el total de columnas se
utilizará el metodo \textbf{shape}

    \begin{Verbatim}[commandchars=\\\{\}]
{\color{incolor}In [{\color{incolor}33}]:} \PY{n}{df}\PY{o}{.}\PY{n}{shape}
\end{Verbatim}


\begin{Verbatim}[commandchars=\\\{\}]
{\color{outcolor}Out[{\color{outcolor}33}]:} (24611, 8)
\end{Verbatim}
            
    Si queremos visualizar los cinco primeros registros para hacernos una
idea de la composición del DataFrame sin tener que cargarlo entero en el
notebook, se utilizará el metodo \textbf{head}

    \begin{Verbatim}[commandchars=\\\{\}]
{\color{incolor}In [{\color{incolor}34}]:} \PY{n}{df}\PY{o}{.}\PY{n}{head}\PY{p}{(}\PY{p}{)}
\end{Verbatim}


\begin{Verbatim}[commandchars=\\\{\}]
{\color{outcolor}Out[{\color{outcolor}34}]:}      state state\_abbreviation   county    fips     party        candidate  \textbackslash{}
         0  Alabama                 AL  Autauga  1001.0  Democrat   Bernie Sanders   
         1  Alabama                 AL  Autauga  1001.0  Democrat  Hillary Clinton   
         2  Alabama                 AL  Baldwin  1003.0  Democrat   Bernie Sanders   
         3  Alabama                 AL  Baldwin  1003.0  Democrat  Hillary Clinton   
         4  Alabama                 AL  Barbour  1005.0  Democrat   Bernie Sanders   
         
            votes  fraction\_votes  
         0    544           0.182  
         1   2387           0.800  
         2   2694           0.329  
         3   5290           0.647  
         4    222           0.078  
\end{Verbatim}
            
    Si por el contrario queremos ver los cinco últimos registros, se
ejecutará el método \textbf{tail}

    \begin{Verbatim}[commandchars=\\\{\}]
{\color{incolor}In [{\color{incolor}35}]:} \PY{n}{df}\PY{o}{.}\PY{n}{tail}\PY{p}{(}\PY{p}{)}
\end{Verbatim}


\begin{Verbatim}[commandchars=\\\{\}]
{\color{outcolor}Out[{\color{outcolor}35}]:}          state state\_abbreviation          county        fips       party  \textbackslash{}
         24606  Wyoming                 WY  Teton-Sublette  95600028.0  Republican   
         24607  Wyoming                 WY   Uinta-Lincoln  95600027.0  Republican   
         24608  Wyoming                 WY   Uinta-Lincoln  95600027.0  Republican   
         24609  Wyoming                 WY   Uinta-Lincoln  95600027.0  Republican   
         24610  Wyoming                 WY   Uinta-Lincoln  95600027.0  Republican   
         
                   candidate  votes  fraction\_votes  
         24606      Ted Cruz      0             0.0  
         24607  Donald Trump      0             0.0  
         24608   John Kasich      0             0.0  
         24609   Marco Rubio      0             0.0  
         24610      Ted Cruz     53             1.0  
\end{Verbatim}
            
    Otro método muy valioso para el entendimiento del DataFrames es el
\textbf{dtypes}. Este metodo nos da información sobre el tipo de datos
que almacena cada columna.

    \begin{Verbatim}[commandchars=\\\{\}]
{\color{incolor}In [{\color{incolor}36}]:} \PY{n}{df}\PY{o}{.}\PY{n}{dtypes}
\end{Verbatim}


\begin{Verbatim}[commandchars=\\\{\}]
{\color{outcolor}Out[{\color{outcolor}36}]:} state                  object
         state\_abbreviation     object
         county                 object
         fips                  float64
         party                  object
         candidate              object
         votes                   int64
         fraction\_votes        float64
         dtype: object
\end{Verbatim}
            
    Aunque si lo que se quiere conseguir es obtener información precisa
acerca de los registros del dataframe se podrá utilizar el metodo
\textbf{describe}

    \begin{Verbatim}[commandchars=\\\{\}]
{\color{incolor}In [{\color{incolor}37}]:} \PY{n}{df}\PY{o}{.}\PY{n}{describe}\PY{p}{(}\PY{p}{)}
\end{Verbatim}


\begin{Verbatim}[commandchars=\\\{\}]
{\color{outcolor}Out[{\color{outcolor}37}]:}                fips          votes  fraction\_votes
         count  2.451100e+04   24611.000000    24611.000000
         mean   2.667152e+07    2306.252773        0.304524
         std    4.200978e+07    9861.183572        0.231401
         min    1.001000e+03       0.000000        0.000000
         25\%    2.109100e+04      68.000000        0.094000
         50\%    4.208100e+04     358.000000        0.273000
         75\%    9.090012e+07    1375.000000        0.479000
         max    9.560004e+07  590502.000000        1.000000
\end{Verbatim}
            
    \subsubsection{Selección: Indexing y
Slicing.}\label{selecciuxf3n-indexing-y-slicing.}

Como las listas en python se puede hacer selección mediante el indexing
y el slicing En este apartado veremos además como seleccionar por
columna o incluso por campo.

Todo dataframe contiene un \textbf{index} que aunque no es
correspondiente a una columna, podemos hacer referencia a el.

    \begin{Verbatim}[commandchars=\\\{\}]
{\color{incolor}In [{\color{incolor}38}]:} \PY{n}{df}\PY{o}{.}\PY{n}{head}\PY{p}{(}\PY{p}{)}
\end{Verbatim}


\begin{Verbatim}[commandchars=\\\{\}]
{\color{outcolor}Out[{\color{outcolor}38}]:}      state state\_abbreviation   county    fips     party        candidate  \textbackslash{}
         0  Alabama                 AL  Autauga  1001.0  Democrat   Bernie Sanders   
         1  Alabama                 AL  Autauga  1001.0  Democrat  Hillary Clinton   
         2  Alabama                 AL  Baldwin  1003.0  Democrat   Bernie Sanders   
         3  Alabama                 AL  Baldwin  1003.0  Democrat  Hillary Clinton   
         4  Alabama                 AL  Barbour  1005.0  Democrat   Bernie Sanders   
         
            votes  fraction\_votes  
         0    544           0.182  
         1   2387           0.800  
         2   2694           0.329  
         3   5290           0.647  
         4    222           0.078  
\end{Verbatim}
            
    \begin{Verbatim}[commandchars=\\\{\}]
{\color{incolor}In [{\color{incolor}39}]:} \PY{c+c1}{\PYZsh{}obtenemos información del index}
         \PY{n+nb}{print}\PY{p}{(}\PY{n}{df}\PY{o}{.}\PY{n}{index}\PY{p}{)}
         
         \PY{c+c1}{\PYZsh{}seleccionamos el registro con index 0}
         \PY{n}{df}\PY{o}{.}\PY{n}{loc}\PY{p}{[}\PY{l+m+mi}{0}\PY{p}{]}
\end{Verbatim}


    \begin{Verbatim}[commandchars=\\\{\}]
RangeIndex(start=0, stop=24611, step=1)

    \end{Verbatim}

\begin{Verbatim}[commandchars=\\\{\}]
{\color{outcolor}Out[{\color{outcolor}39}]:} state                        Alabama
         state\_abbreviation                AL
         county                       Autauga
         fips                            1001
         party                       Democrat
         candidate             Bernie Sanders
         votes                            544
         fraction\_votes                 0.182
         Name: 0, dtype: object
\end{Verbatim}
            
    El index es un puntero que hace referencia al orden en el dataframe.
Este puntero e puede cambiar a cualquier otra columna:

    \begin{Verbatim}[commandchars=\\\{\}]
{\color{incolor}In [{\color{incolor}40}]:} \PY{n}{df2} \PY{o}{=} \PY{n}{df}\PY{o}{.}\PY{n}{set\PYZus{}index}\PY{p}{(}\PY{l+s+s2}{\PYZdq{}}\PY{l+s+s2}{county}\PY{l+s+s2}{\PYZdq{}}\PY{p}{)}
\end{Verbatim}


    \begin{Verbatim}[commandchars=\\\{\}]
{\color{incolor}In [{\color{incolor}41}]:} \PY{n}{df2}\PY{o}{.}\PY{n}{head}\PY{p}{(}\PY{p}{)}
\end{Verbatim}


\begin{Verbatim}[commandchars=\\\{\}]
{\color{outcolor}Out[{\color{outcolor}41}]:}            state state\_abbreviation    fips     party        candidate  votes  \textbackslash{}
         county                                                                          
         Autauga  Alabama                 AL  1001.0  Democrat   Bernie Sanders    544   
         Autauga  Alabama                 AL  1001.0  Democrat  Hillary Clinton   2387   
         Baldwin  Alabama                 AL  1003.0  Democrat   Bernie Sanders   2694   
         Baldwin  Alabama                 AL  1003.0  Democrat  Hillary Clinton   5290   
         Barbour  Alabama                 AL  1005.0  Democrat   Bernie Sanders    222   
         
                  fraction\_votes  
         county                   
         Autauga           0.182  
         Autauga           0.800  
         Baldwin           0.329  
         Baldwin           0.647  
         Barbour           0.078  
\end{Verbatim}
            
    Como podemos comprobar ahora la columna \emph{county} será referenciado
como index.

    \begin{Verbatim}[commandchars=\\\{\}]
{\color{incolor}In [{\color{incolor}42}]:} \PY{n}{df2}\PY{o}{.}\PY{n}{index}
\end{Verbatim}


\begin{Verbatim}[commandchars=\\\{\}]
{\color{outcolor}Out[{\color{outcolor}42}]:} Index(['Autauga', 'Autauga', 'Baldwin', 'Baldwin', 'Barbour', 'Barbour',
                'Bibb', 'Bibb', 'Blount', 'Blount',
                {\ldots}
                'Sweetwater-Carbon', 'Sweetwater-Carbon', 'Teton-Sublette',
                'Teton-Sublette', 'Teton-Sublette', 'Teton-Sublette', 'Uinta-Lincoln',
                'Uinta-Lincoln', 'Uinta-Lincoln', 'Uinta-Lincoln'],
               dtype='object', name='county', length=24611)
\end{Verbatim}
            
    Ahora que se ha cambiado el index, se puede seleccionar por condado:

    \begin{Verbatim}[commandchars=\\\{\}]
{\color{incolor}In [{\color{incolor}43}]:} \PY{n}{df2}\PY{o}{.}\PY{n}{loc}\PY{p}{[}\PY{l+s+s2}{\PYZdq{}}\PY{l+s+s2}{Los Angeles}\PY{l+s+s2}{\PYZdq{}}\PY{p}{]}
\end{Verbatim}


\begin{Verbatim}[commandchars=\\\{\}]
{\color{outcolor}Out[{\color{outcolor}43}]:}                   state state\_abbreviation    fips       party  \textbackslash{}
         county                                                           
         Los Angeles  California                 CA  6037.0    Democrat   
         Los Angeles  California                 CA  6037.0    Democrat   
         Los Angeles  California                 CA  6037.0  Republican   
         Los Angeles  California                 CA  6037.0  Republican   
         Los Angeles  California                 CA  6037.0  Republican   
         
                            candidate   votes  fraction\_votes  
         county                                                
         Los Angeles   Bernie Sanders  434656           0.420  
         Los Angeles  Hillary Clinton  590502           0.570  
         Los Angeles     Donald Trump  179130           0.698  
         Los Angeles      John Kasich   33559           0.131  
         Los Angeles         Ted Cruz   30775           0.120  
\end{Verbatim}
            
    Con esto demostramos que \textbf{loc} selecciona por indice no por
posición. Si lo que queremos por el contrario es seleccionar el número
de fila en lugar del indice se deberá utilizar el método \textbf{iloc}

    \begin{Verbatim}[commandchars=\\\{\}]
{\color{incolor}In [{\color{incolor}44}]:} \PY{n}{df2}\PY{o}{.}\PY{n}{iloc}\PY{p}{[}\PY{l+m+mi}{0}\PY{p}{]}
\end{Verbatim}


\begin{Verbatim}[commandchars=\\\{\}]
{\color{outcolor}Out[{\color{outcolor}44}]:} state                        Alabama
         state\_abbreviation                AL
         fips                            1001
         party                       Democrat
         candidate             Bernie Sanders
         votes                            544
         fraction\_votes                 0.182
         Name: Autauga, dtype: object
\end{Verbatim}
            
    Los dataframes soportan parametros de busqueda entre corchetes como los
diccionarios de Python:

    \begin{Verbatim}[commandchars=\\\{\}]
{\color{incolor}In [{\color{incolor}45}]:} \PY{n}{df}\PY{p}{[}\PY{l+s+s2}{\PYZdq{}}\PY{l+s+s2}{state}\PY{l+s+s2}{\PYZdq{}}\PY{p}{]}\PY{p}{[}\PY{p}{:}\PY{l+m+mi}{10}\PY{p}{]}
\end{Verbatim}


\begin{Verbatim}[commandchars=\\\{\}]
{\color{outcolor}Out[{\color{outcolor}45}]:} 0    Alabama
         1    Alabama
         2    Alabama
         3    Alabama
         4    Alabama
         5    Alabama
         6    Alabama
         7    Alabama
         8    Alabama
         9    Alabama
         Name: state, dtype: object
\end{Verbatim}
            
    Saber esto es muy util, ya que nos permite acceder al contenido de las
columnas. En este ejemplo se introducirá una nueva columna y se asignará
como valor para esa columna el numero 1.

    \begin{Verbatim}[commandchars=\\\{\}]
{\color{incolor}In [{\color{incolor}46}]:} \PY{n}{df}\PY{p}{[}\PY{l+s+s2}{\PYZdq{}}\PY{l+s+s2}{shape}\PY{l+s+s2}{\PYZdq{}}\PY{p}{]} \PY{o}{=} \PY{l+m+mi}{1}
         \PY{n}{df}\PY{o}{.}\PY{n}{head}\PY{p}{(}\PY{p}{)}
\end{Verbatim}


\begin{Verbatim}[commandchars=\\\{\}]
{\color{outcolor}Out[{\color{outcolor}46}]:}      state state\_abbreviation   county    fips     party        candidate  \textbackslash{}
         0  Alabama                 AL  Autauga  1001.0  Democrat   Bernie Sanders   
         1  Alabama                 AL  Autauga  1001.0  Democrat  Hillary Clinton   
         2  Alabama                 AL  Baldwin  1003.0  Democrat   Bernie Sanders   
         3  Alabama                 AL  Baldwin  1003.0  Democrat  Hillary Clinton   
         4  Alabama                 AL  Barbour  1005.0  Democrat   Bernie Sanders   
         
            votes  fraction\_votes  shape  
         0    544           0.182      1  
         1   2387           0.800      1  
         2   2694           0.329      1  
         3   5290           0.647      1  
         4    222           0.078      1  
\end{Verbatim}
            
    Si seleccionamos una columna, obtenemos una Serie, si seleccionamos dos
o más, obtenemos un dataframe.

    \begin{Verbatim}[commandchars=\\\{\}]
{\color{incolor}In [{\color{incolor}47}]:} \PY{n+nb}{type}\PY{p}{(}\PY{n}{df}\PY{p}{[}\PY{l+s+s1}{\PYZsq{}}\PY{l+s+s1}{county}\PY{l+s+s1}{\PYZsq{}}\PY{p}{]}\PY{p}{)}
\end{Verbatim}


\begin{Verbatim}[commandchars=\\\{\}]
{\color{outcolor}Out[{\color{outcolor}47}]:} pandas.core.series.Series
\end{Verbatim}
            
    \begin{Verbatim}[commandchars=\\\{\}]
{\color{incolor}In [{\color{incolor}48}]:} \PY{n+nb}{type}\PY{p}{(}\PY{n}{df}\PY{p}{[}\PY{p}{[}\PY{l+s+s2}{\PYZdq{}}\PY{l+s+s2}{county}\PY{l+s+s2}{\PYZdq{}}\PY{p}{,} \PY{l+s+s2}{\PYZdq{}}\PY{l+s+s2}{candidate}\PY{l+s+s2}{\PYZdq{}}\PY{p}{]}\PY{p}{]}\PY{p}{)}
\end{Verbatim}


\begin{Verbatim}[commandchars=\\\{\}]
{\color{outcolor}Out[{\color{outcolor}48}]:} pandas.core.frame.DataFrame
\end{Verbatim}
            
    Se puede además filtrar un dataframe de la misma forma que se filtra en
numpy. Además estas condiciones se pueden concatenar utilizano el
operador \textbf{\&}

    \begin{Verbatim}[commandchars=\\\{\}]
{\color{incolor}In [{\color{incolor}49}]:} \PY{n}{df}\PY{p}{[}\PY{n}{df}\PY{o}{.}\PY{n}{votes}\PY{o}{\PYZgt{}}\PY{o}{=}\PY{l+m+mi}{590502}\PY{p}{]}
\end{Verbatim}


\begin{Verbatim}[commandchars=\\\{\}]
{\color{outcolor}Out[{\color{outcolor}49}]:}            state state\_abbreviation       county    fips     party  \textbackslash{}
         1386  California                 CA  Los Angeles  6037.0  Democrat   
         
                     candidate   votes  fraction\_votes  shape  
         1386  Hillary Clinton  590502            0.57      1  
\end{Verbatim}
            
    \begin{Verbatim}[commandchars=\\\{\}]
{\color{incolor}In [{\color{incolor}50}]:} \PY{n}{df}\PY{p}{[}\PY{p}{(}\PY{n}{df}\PY{o}{.}\PY{n}{county}\PY{o}{==}\PY{l+s+s2}{\PYZdq{}}\PY{l+s+s2}{Manhattan}\PY{l+s+s2}{\PYZdq{}}\PY{p}{)} \PY{o}{\PYZam{}} \PY{p}{(}\PY{n}{df}\PY{o}{.}\PY{n}{party}\PY{o}{==}\PY{l+s+s2}{\PYZdq{}}\PY{l+s+s2}{Democrat}\PY{l+s+s2}{\PYZdq{}}\PY{p}{)}\PY{p}{]}
\end{Verbatim}


\begin{Verbatim}[commandchars=\\\{\}]
{\color{outcolor}Out[{\color{outcolor}50}]:}           state state\_abbreviation     county     fips     party  \textbackslash{}
         15011  New York                 NY  Manhattan  36061.0  Democrat   
         15012  New York                 NY  Manhattan  36061.0  Democrat   
         
                      candidate   votes  fraction\_votes  shape  
         15011   Bernie Sanders   90227           0.337      1  
         15012  Hillary Clinton  177496           0.663      1  
\end{Verbatim}
            
    Otro metodo muy utilizado para la selección de registros de un dataframe
es el método \textbf{query} el cual nos permite hacer referencias al
contenido de otras variables mediante el operador **@**.

    \begin{Verbatim}[commandchars=\\\{\}]
{\color{incolor}In [{\color{incolor}51}]:} \PY{n}{county} \PY{o}{=} \PY{l+s+s2}{\PYZdq{}}\PY{l+s+s2}{Manhattan}\PY{l+s+s2}{\PYZdq{}}
         \PY{n}{df}\PY{o}{.}\PY{n}{query}\PY{p}{(}\PY{l+s+s2}{\PYZdq{}}\PY{l+s+s2}{county==@county}\PY{l+s+s2}{\PYZdq{}}\PY{p}{)}
\end{Verbatim}


\begin{Verbatim}[commandchars=\\\{\}]
{\color{outcolor}Out[{\color{outcolor}51}]:}           state state\_abbreviation     county     fips       party  \textbackslash{}
         15011  New York                 NY  Manhattan  36061.0    Democrat   
         15012  New York                 NY  Manhattan  36061.0    Democrat   
         15162  New York                 NY  Manhattan  36061.0  Republican   
         15163  New York                 NY  Manhattan  36061.0  Republican   
         15164  New York                 NY  Manhattan  36061.0  Republican   
         
                      candidate   votes  fraction\_votes  shape  
         15011   Bernie Sanders   90227           0.337      1  
         15012  Hillary Clinton  177496           0.663      1  
         15162     Donald Trump   10393           0.418      1  
         15163      John Kasich   11251           0.452      1  
         15164         Ted Cruz    3243           0.130      1  
\end{Verbatim}
            
    \subsection{Procesado de Dataframes.}\label{procesado-de-dataframes.}

En este apartado se observarán los métodos más relevantes para procesar
DataFrames.

Para ordenar un DataFrame se utilizará el método \textbf{sort\_values},
el cual ordenará en función al valor de la columna que recibe como
parámetro. Además como segundo parámetro se le puede ordenar que los
ordene ascendente o descendentemente.

    \begin{Verbatim}[commandchars=\\\{\}]
{\color{incolor}In [{\color{incolor}52}]:} \PY{n}{df\PYZus{}sorted} \PY{o}{=} \PY{n}{df}\PY{o}{.}\PY{n}{sort\PYZus{}values}\PY{p}{(}\PY{n}{by}\PY{o}{=}\PY{l+s+s2}{\PYZdq{}}\PY{l+s+s2}{votes}\PY{l+s+s2}{\PYZdq{}}\PY{p}{,} \PY{n}{ascending}\PY{o}{=}\PY{k+kc}{False}\PY{p}{)}
         \PY{n}{df\PYZus{}sorted}\PY{o}{.}\PY{n}{head}\PY{p}{(}\PY{p}{)}
\end{Verbatim}


\begin{Verbatim}[commandchars=\\\{\}]
{\color{outcolor}Out[{\color{outcolor}52}]:}            state state\_abbreviation        county        fips     party  \textbackslash{}
         1386  California                 CA   Los Angeles      6037.0  Democrat   
         1385  California                 CA   Los Angeles      6037.0  Democrat   
         4451    Illinois                 IL       Chicago  91700103.0  Democrat   
         4450    Illinois                 IL       Chicago  91700103.0  Democrat   
         4463    Illinois                 IL  Cook Suburbs  91700104.0  Democrat   
         
                     candidate   votes  fraction\_votes  shape  
         1386  Hillary Clinton  590502           0.570      1  
         1385   Bernie Sanders  434656           0.420      1  
         4451  Hillary Clinton  366954           0.536      1  
         4450   Bernie Sanders  311225           0.454      1  
         4463  Hillary Clinton  249217           0.536      1  
\end{Verbatim}
            
    Un método que nos permite agrupar columnas es el método
\textbf{groupby}. Utilizaremos este método para agrupar las columnas
referentes al estado y la referente al partido político del actual
dataframe. Posteriormente realizaremos una selección de la suma de sus
votos.

Con esta operación obtendremos una lista con los resultados de voto por
partido en los distintos estados de Estados Unidos.

    \begin{Verbatim}[commandchars=\\\{\}]
{\color{incolor}In [{\color{incolor}53}]:} \PY{n}{df}\PY{o}{.}\PY{n}{groupby}\PY{p}{(}\PY{p}{[}\PY{l+s+s2}{\PYZdq{}}\PY{l+s+s2}{state}\PY{l+s+s2}{\PYZdq{}}\PY{p}{,}\PY{l+s+s2}{\PYZdq{}}\PY{l+s+s2}{party}\PY{l+s+s2}{\PYZdq{}}\PY{p}{]}\PY{p}{)}\PY{p}{[}\PY{l+s+s2}{\PYZdq{}}\PY{l+s+s2}{votes}\PY{l+s+s2}{\PYZdq{}}\PY{p}{]}\PY{o}{.}\PY{n}{sum}\PY{p}{(}\PY{p}{)}
\end{Verbatim}


\begin{Verbatim}[commandchars=\\\{\}]
{\color{outcolor}Out[{\color{outcolor}53}]:} state           party     
         Alabama         Democrat       386327
                         Republican     837632
         Alaska          Democrat          539
                         Republican      21930
         Arizona         Democrat       399097
                         Republican     435103
         Arkansas        Democrat       209448
                         Republican     396523
         California      Democrat      3442623
                         Republican    1495574
         Colorado        Democrat       121184
         Connecticut     Democrat       322485
                         Republican     208817
         Delaware        Democrat        92609
                         Republican      67807
         Florida         Democrat      1664003
                         Republican    2276926
         Georgia         Democrat       757340
                         Republican    1275601
         Hawaii          Democrat        33658
                         Republican      13228
         Idaho           Democrat        23705
                         Republican     215284
         Illinois        Democrat      1987834
                         Republican    1384703
         Indiana         Democrat       638638
                         Republican    1080653
         Iowa            Democrat       139980
                         Republican     186724
         Kansas          Democrat        39043
                                        {\ldots}   
         Oklahoma        Democrat       313392
                         Republican     452731
         Oregon          Democrat       572485
                         Republican     361490
         Pennsylvania    Democrat      1638644
                         Republican    1537696
         Rhode Island    Democrat       119213
                         Republican      60381
         South Carolina  Democrat       367491
                         Republican     737917
         South Dakota    Democrat        53004
                         Republican      66877
         Tennessee       Democrat       365637
                         Republican     834939
         Texas           Democrat      1410641
                         Republican    2737248
         Utah            Democrat        76999
                         Republican     177204
         Vermont         Democrat       134198
                         Republican      58762
         Virginia        Democrat       778865
                         Republican    1012807
         Washington      Democrat        26299
                         Republican     510851
         West Virginia   Democrat       210214
                         Republican     188138
         Wisconsin       Democrat      1000703
                         Republican    1072699
         Wyoming         Democrat          280
                         Republican        903
         Name: votes, Length: 95, dtype: int64
\end{Verbatim}
            
    Mediante la función \textbf{apply} al que se puede agregar valores a una
columna a través de los resultados de una función

    \begin{Verbatim}[commandchars=\\\{\}]
{\color{incolor}In [{\color{incolor}54}]:} \PY{c+c1}{\PYZsh{}mediante esta función obtenemos la primera letra de cada estado}
         \PY{n}{df}\PY{o}{.}\PY{n}{state\PYZus{}abbreviation}\PY{o}{.}\PY{n}{apply}\PY{p}{(}\PY{k}{lambda} \PY{n}{s}\PY{p}{:}\PY{n}{s}\PY{p}{[}\PY{l+m+mi}{0}\PY{p}{]}\PY{p}{)}
         
         \PY{c+c1}{\PYZsh{}si esto lo agregamos a una columa podemos volcarlo al DataFrame}
         \PY{n}{df}\PY{p}{[}\PY{l+s+s2}{\PYZdq{}}\PY{l+s+s2}{letra\PYZus{}estado}\PY{l+s+s2}{\PYZdq{}}\PY{p}{]} \PY{o}{=} \PY{n}{df}\PY{o}{.}\PY{n}{state\PYZus{}abbreviation}\PY{o}{.}\PY{n}{apply}\PY{p}{(}\PY{k}{lambda} \PY{n}{s}\PY{p}{:} \PY{n}{s}\PY{p}{[}\PY{l+m+mi}{0}\PY{p}{]}\PY{p}{)}
         \PY{n}{df}\PY{o}{.}\PY{n}{head}\PY{p}{(}\PY{p}{)}
\end{Verbatim}


\begin{Verbatim}[commandchars=\\\{\}]
{\color{outcolor}Out[{\color{outcolor}54}]:}      state state\_abbreviation   county    fips     party        candidate  \textbackslash{}
         0  Alabama                 AL  Autauga  1001.0  Democrat   Bernie Sanders   
         1  Alabama                 AL  Autauga  1001.0  Democrat  Hillary Clinton   
         2  Alabama                 AL  Baldwin  1003.0  Democrat   Bernie Sanders   
         3  Alabama                 AL  Baldwin  1003.0  Democrat  Hillary Clinton   
         4  Alabama                 AL  Barbour  1005.0  Democrat   Bernie Sanders   
         
            votes  fraction\_votes  shape letra\_estado  
         0    544           0.182      1            A  
         1   2387           0.800      1            A  
         2   2694           0.329      1            A  
         3   5290           0.647      1            A  
         4    222           0.078      1            A  
\end{Verbatim}
            
    \subsubsection{Exportar/Importar DataFrame a
excel}\label{exportarimportar-dataframe-a-excel}

Además de exportar e importar ficheros csv también podemos exportar e
importar de excel, pero para ello será necesario instalar el paquete
\emph{xlwt}

\begin{Shaded}
\begin{Highlighting}[]
\NormalTok{!}\ExtensionTok{conda}\NormalTok{ install -y xlwt}
\end{Highlighting}
\end{Shaded}

    \begin{Verbatim}[commandchars=\\\{\}]
{\color{incolor}In [{\color{incolor}55}]:} \PY{n}{rick\PYZus{}and\PYZus{}morty}\PY{o}{.}\PY{n}{to\PYZus{}excel}\PY{p}{(}\PY{l+s+s2}{\PYZdq{}}\PY{l+s+s2}{rick\PYZus{}y\PYZus{}morty.xls}\PY{l+s+s2}{\PYZdq{}}\PY{p}{,} \PY{n}{sheet\PYZus{}name}\PY{o}{=}\PY{l+s+s2}{\PYZdq{}}\PY{l+s+s2}{personajes}\PY{l+s+s2}{\PYZdq{}}\PY{p}{)}
\end{Verbatim}


    \begin{Verbatim}[commandchars=\\\{\}]
{\color{incolor}In [{\color{incolor}56}]:} \PY{n}{rick\PYZus{}morty2} \PY{o}{=} \PY{n}{pd}\PY{o}{.}\PY{n}{read\PYZus{}excel}\PY{p}{(}\PY{l+s+s2}{\PYZdq{}}\PY{l+s+s2}{rick\PYZus{}y\PYZus{}morty.xls}\PY{l+s+s2}{\PYZdq{}}\PY{p}{,} \PY{n}{sheet\PYZus{}name}\PY{o}{=}\PY{l+s+s2}{\PYZdq{}}\PY{l+s+s2}{personajes}\PY{l+s+s2}{\PYZdq{}}\PY{p}{)}
\end{Verbatim}


    \begin{Verbatim}[commandchars=\\\{\}]
{\color{incolor}In [{\color{incolor}57}]:} \PY{n}{rick\PYZus{}morty2}\PY{o}{.}\PY{n}{head}\PY{p}{(}\PY{p}{)}
\end{Verbatim}


\begin{Verbatim}[commandchars=\\\{\}]
{\color{outcolor}Out[{\color{outcolor}57}]:}    Unnamed: 0 nombre apellidos  edad
         0           0   Rick   Sanchez    60
         1           1  Morty     Smith    14
\end{Verbatim}
            

    % Add a bibliography block to the postdoc
    
    
    
    \end{document}
